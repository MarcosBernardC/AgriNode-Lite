% ==========================================================
% 🧩 Preambulo principal de AgriNode-Lite (LuaLaTeX)
% ==========================================================

% Clase de documento: informe técnico base
\documentclass[12pt,a4paper]{article}

% ----------------------------------------------------------
% Paquetes esenciales del motor LuaLaTeX
% ----------------------------------------------------------
\usepackage{fontspec}      % Fuentes del sistema
\usepackage{microtype}     % Microtipografía avanzada
\usepackage{xcolor}        % Colores
\usepackage{graphicx}      % Imágenes
\usepackage{geometry}      % Márgenes y dimensiones
\usepackage{hyperref}      % Enlaces y referencias
\usepackage{minted}        % Resaltado de código (requiere -shell-escape)
\usepackage{fancyhdr}      % Encabezados y pies de página
\usepackage{setspace}      % Espaciado entre líneas
\usepackage[main=spanish,provide=*]{babel}  % Idioma español
\usepackage{csquotes}      % Comillas tipográficas
\usepackage{booktabs}      % Tablas profesionales
\usepackage{siunitx}       % Unidades y notación científica

% ----------------------------------------------------------
% Configuración de página y formato
% ----------------------------------------------------------
\geometry{
  a4paper,
  top=2.5cm,
  bottom=2.5cm,
  left=2.5cm,
  right=2.5cm
}

\setstretch{1.15}  % Interlineado agradable

% ----------------------------------------------------------
% Fuentes del sistema (recomendadas en Garuda Linux)
% ----------------------------------------------------------
% Para soporte completo Unicode y emoji 🌿
\setmainfont{Noto Serif}
\setsansfont{Noto Sans}
\setmonofont{JetBrains Mono}

% Fuente separada para emojis (usada en encabezados, etc.)
\newfontfamily\emojifont{Noto Color Emoji}

\fancyhead[L]{\textbf{AgriNode-Lite} \raisebox{-0.1em}{\emojifont 🌾}}


% Ejemplo de cómo mezclar emoji dentro del texto o encabezado:
% \fancyhead[L]{\textbf{AgriNode-Lite} \raisebox{-0.1em}{\emojifont 🌿}}


% ----------------------------------------------------------
% Colores y estilo de código fuente
% ----------------------------------------------------------
\definecolor{agrigreen}{HTML}{6CA36C}
\definecolor{agricyan}{HTML}{00A8A8}
\definecolor{agriorange}{HTML}{E69826}

\setminted{
  fontsize=\footnotesize,
  frame=lines,
  framesep=3pt,
  bgcolor=gray!5,
  breaklines,
  tabsize=2,
  linenos,
  numbersep=6pt,
  xleftmargin=1.5em,
  style=friendly
}

% ----------------------------------------------------------
% Enlaces e hipervínculos
% ----------------------------------------------------------
\hypersetup{
  colorlinks=true,
  linkcolor=agrigreen,
  urlcolor=agricyan,
  citecolor=agriorange,
  pdftitle={AgriNode-Lite - Informe Técnico},
  pdfauthor={Bernard},
  pdfsubject={Proyecto AgriNode-Lite - Fase 1},
  pdfcreator={LuaLaTeX}
}

% ----------------------------------------------------------
% Encabezado y pie de página
% ----------------------------------------------------------
\pagestyle{fancy}
\fancyhf{}
\fancyhead[L]{\textbf{AgriNode-Lite }}
\fancyhead[R]{\leftmark}
\fancyfoot[C]{\thepage}

% ----------------------------------------------------------
% Inicio del documento
% ----------------------------------------------------------
